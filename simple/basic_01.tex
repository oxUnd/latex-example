\documentclass[11pt]{article}

% font include CJK
\usepackage{xltxtra}

\usepackage{listings}

\newfontfamily\Hei{SimHei}
\newfontfamily\Song{FangSong}
\newfontfamily\Code{Consolas}

\title{FIS2.z}
\author{Xiang Shouding}

\begin{document}

{\Code \maketitle}

\section{\Hei 背景}
    \subsection{\Hei 介绍}
    % all \Song
        \Song
        {\Code FIS2 }推出大概有一年半左右,成绩斐然,具体数据就不详说了。在高速发展的\\
        背后不免出现这样那样的不适,所以通过对 {\Code FIS }工具改进来更符合前端工程。\\
        
    \subsection{\Hei 问题}
        这块主要罗列一下现在 {\Code FIS2 }的一些一直被诟病的问题\\
        \begin{itemize}
            \item {\Code `fis.config.merge` }覆盖问题
            \item 插件点不支持异步({ \Code node.js }这点真挫)
            \item 配置文件{ \Code modules }和{ \Code settings }分开,不直观的问题;
            \item 若干{ \Code issuess }
        \end{itemize}
        
\section{\Hei 概要设计}

% -- code start --
\Code
\begin{lstlisting}[language=c]
    #include <stdio.h>
    int main() {
        printf("Hello,World!\n");
        return 0;
    }
\end{lstlisting}

% -- code end --

\Hei

\end{document}
